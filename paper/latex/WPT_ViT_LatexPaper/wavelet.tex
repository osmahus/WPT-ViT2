\documentclass{svproc}
\usepackage{graphicx} 
\usepackage{amsmath}
\usepackage{biblatex}
\addbibresource{lit.bib}

\title{Using Wavelet Packets with Vision Transformer for Deep Fake Detection}
\author{Osama Rawy, AbdulRahman AlTahhan}
\institute{University of Leeds, School of Computing, ODL MSc in AI, UK.}
\date{June 2024}

\begin{document}
\maketitle

\begin{abstract}
    This paper proposes a new method for combining wavelet packet transform with vision transformer to create a binary classifier for image deep-fake detection. We showed that we can achieve comparable accuracy with previous work in this research area, using smaller model sizes and lower GPU and CPU requirements. We tested our model using the CIFAKE dataset, and eventually, we put the code and the model for open access on the following URL:  
    
    \keywords{Deep Fake Detection, vision Transformer, Wavelet Packet Transform.}
\end{abstract}

\section{Introduction}
    This is the introduction to your project. This is nice intro. In \cite{NuclearPlant}
    
\section{Literature Review}
    \cite{TD0-Replay} has established a new TD(0) method that replays all past experiences. On the hand, \cite{TD-Replay} has taken this further to include a target that incorporates all past updates via TD($\lambda$). \cite{ConjugateTD} applied conjugate gradient update on TD.


\section{Methodology}
    In this section, we lay out the methodology. Note how in Fig. \ref{fig:my_label} we have shown the boundary.


    \begin{figure}
        \centering
        \includegraphics[scale=.75]{figures/DecisionBoundary.png}
        \caption{Caption}
        \label{fig:fig2}
    \end{figure}

    
    \begin{figure}
        \centering
        \includegraphics[scale=.75]{figures/DecisionBoundary.png}
        \caption{Caption}
        \label{fig:my_label}
    \end{figure}


    
    In Fig. \ref{fig:my_label} that $\alpha = n^2$. In eq. (\ref{eq:sum_i}) we have shown that the $ 1+2+3+4 = 4\times 5 /2=10$.
    \begin{align*}
        \sum_{i=1}^{n} y &= 10 \\
        M &= \beta ^2 \\
        \boldsymbol{B}^\top &= \boldsymbol{\Lambda}^2 \\
    \end{align*}
    
    \begin{align}
        \label{eq:sum_i}
        \sum_{i=1}^n i &= \frac{(n+1)n}{2} \\  
        \nonumber
        \beta ^2 &= \alpha_i^n
    \end{align}

\section{Experiment Results}
    \subsection{Experiment 1}
        \subsubsection{Exp}
    Note that the figure and the tables might be laid out on another page. Do not worry about that, and do not attempt to change it. Leave this to Latex.
    
    
    \begin{table}
        \caption{This is a table}
        \begin{center}
            \begin{tabular}{rlc}
                \hline
                \multicolumn{1}{l}{Year}&\multicolumn{1}{l}{World}&\multicolumn{1}{l}{Duration}\\
                \hline
                8000 B.C.  &     5,000,000 &  10\\
                  50 A.D.  &   200,000,000 &  20\\
                1650 A.D.  &   500,000,000 &  30\\
                \hline
            \end{tabular}
        \end{center}
    \end{table}

\section{Conclusion and Future Work}
We have conducted a study on ...

\printbibliography 
\end{document}
